\section{Background--the human heart }

Using Figure~\ref{fig:heartOverview}, we describe the
electrical conduction system of the heart.
% \noindent \textbf{Heart.}
The human heart (see Figure~\ref{fig:heart}) is a biological pump that
is essential for blood circulation using its rhythmic pumping activity.
It achieves this by regularly contracting and relaxing its muscles,
which is orchestrated through flow of electrical signals along a set of
cellular groups called nodes and associated paths called its
\emph{conduction system}. The source of the signal is a natural pacemaker in the left ventricle, called the \ac{SA} node, which triggers
automatically. First, this signal travels through the left and right
atria, contracting the muscles and pushing the blood into the
ventricles.

Second, to ensure both ventricles are filled, the \ac{AV} node
introduces critical delay in the conduction system. Finally, the
electrical signal propagates through both ventricles. This contracts the
muscles and pumps the blood out of the heart.

\noindent \textbf{\acf{AP}.} The propagation of electrical signals at
the cellular / nodal level is described as a change in voltage across
the cell membrane due to ionic flow. Over a period of time, this change
is depicted as the \acf{AP}~\cite{chen14} (see 
Figure~\ref{fig:actionPotential}). It can be described in four
stages.
\begin{enumerate}
\item \acf{RP}: This is the steady state, when the node is awaiting
  activation by an external stimulus.
\item \acf{ST}: When the external stimulus is above a threshold voltage ($V_T$).
\item \acf{UP}: After continuous stimulation, the cell's voltage 
	reaches the threshold voltage ($V_T$. This causes   the node to
  depolarises (inward flow of positive ions) and contracts the
  muscles. This depolarisation yields a stimulus that activates
  neighbouring nodes.
\item \acf{ERP}: Once activated, the node cannot be activated again due
  to the recovery process of the ionic channels. Any new stimulus will
  be blocked during this refractory period.
\end{enumerate}
Finally, the ionic channels will partially
recover (\acf{RRP}) and then fully recover (\ac{RP}). 
If activated again by
external stimulus during \ac{RRP}, the morphology
of the \ac{AP} will be shorter when compared to \ac{RP}.

\noindent \textbf{Abstract model.} The human heart has over two trillion
cells. For analytical purposes, a model consisting of a network of $33$~nodes 
is presented in literature and is used for testing off-the-shelf
pacemakers~\cite{zhihao12,chen14}. The abstracted electrical conduction
system consisting of nodes and paths is presented in
Figure~\ref{fig:heartNetwork}. The functionality of each node is
described using \acf{HIOA} in Section~\ref{sec:HA}.  During implementation, the 
connections between nodes are implemented as buffers where the length of the 
buffer depends on the conduction delay and step size.  Later in 
Section~\ref{sec:codeGen}, we use this network of nodes to illustrate our 
modular code generation.



%%% Local Variables:
%%% mode: latex
%%% TeX-master: "../DATE2016_codegen"
%%% End:
