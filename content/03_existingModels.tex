\section{Existing Heart models }
\label{sec:existingHeartModels}

In this section we discuss existing models 
of the heart used for pacemaker verifications.

\begin{figure*}
	{
	\centering
	
	\framebox[0.45\textwidth]{
		\begin{minipage}{0.45\textwidth}
		\subfigure[\label{fig:heart} Diagram of the heart. Reproduced 
		from~\cite{zhihao12}]{
				
\begin{tikzpicture}[->,>=stealth',shorten >=1pt,auto,
node distance=4cm,
semithick,scale=0.8, transform shape]
\tikzstyle{every state}=[rectangle,rounded corners,
minimum height = 1.9cm, text width=2.35cm, text centered, fill=red!20,draw=none,text=black, draw,line width=0.3mm]
\tikzset{
	atrialCell/.style={
		fill=green!20,
		circle,
		minimum size=1.5cm,
		draw,
		line width=0.2mm
	}
}
\tikzset{
	path/.style={
		fill=green!20,
		rectangle,
		minimum size=1.5cm,
		draw,
		line width=0.2mm
	}
}



\node[state, 
label={[shift={(0,0)}]$t<=Trest\_Max$}, 
label={[shift={(-1.5,-2.5)}]\large $ \mathbf{q_0}$ }]
(Q0)  {$\dot{t}=1$};

%entry
\path[<-, dashed] (Q0.130) edge node[below, align=left, shift={(0.5,1)}] {
	\footnotesize initial \\
	\footnotesize $t=0$
} ++(0cm,1.25cm);




\node[state, 
label={[shift={(0,0)}]$URGENT$}, 
label={[shift={(-1.5,-2.5)}]\large $\mathbf{q_1}$   }]
(Q1) [node distance=6.3cm, right of=Q0] {$\dot{t}=1$};

\node[state, 
label={[shift={(0.2,-2.5)}]$t<=Terp\_Max$}, 
label={[shift={(-1.5,-2.5)}]\large $\mathbf{q_2}$  }] 
(Q2) [below of=Q1] {$\dot{t}=1$};

\node[state, 
label={[shift={(0,-2.5)}]$t<=Trrp\_Max$}, 
label={[shift={(-1.5,-2.5)}]\large $\mathbf{q_3}$  }]
(Q3) [below of=Q0] {$\dot{t}=1$};

\path[->] (Q0.-20) edge node[align=center] {
	$\frac{Act\_node?}{t=0}$
} (Q1.200);

\path[->] (Q0.20) edge node[align=center] {
	$\frac{t>Trest\_min?}{t=0}$
} (Q1.160);


\path[->] (Q1) edge node[align=center] {
	$\frac{Act\_path!}{}$
} (Q2);

\path[->] (Q2) edge node[align=center] {
	$\frac{t>=Terp_min}{t=0}$
} (Q3);

\path[->] (Q3) edge node[align=center,shift={(1.8,0)}] {
	$\frac{t>=Trrp_min}{t=0}$
} (Q0);


\node[ draw,   inner xsep=1.1cm, inner ysep=1.9cm, shift={(0, 0.35)},
label={[label distance=0cm]60:~}, 
fit= (Q0)(Q2)] (BOX) {};

%--------------------------- nodes on top ------------
\node[atrialCell, above of = Q0, xshift=-0.5cm, node distance = 5cm,
label={[shift={(1.8,-0.4)}]\large $Act\_path$ },
label={[shift={(1.8,-1.2)}]\large $Act\_node1$ }]
(AN1){$N_1$};


\node[path, right of = AN1, node distance = 3.7cm] 
(P1){$Path$};

\node[atrialCell, right of = P1, node distance = 3.7cm,
label={[shift={(-1.8,-0.4)}]\large $Act\_path$ },
label={[shift={(-1.8,-1.2)}]\large $Act\_node2$ }] 
(AN2){$N_2$};

\draw[<-](P1.160) -- (P1.160-|AN1.east);
\draw[->](P1.210) -- (P1.210-|AN1.east);

\draw[<-](P1.20) -- (P1.20-|AN2.west);
\draw[->](P1.330) -- (P1.330-|AN2.west);



\draw[-,dashed](AN1.-45) -- (BOX.north east);
\draw[-,dashed](AN1.230) -- (BOX.north west);




\end{tikzpicture}
		}
		
		\subfigure[The four stages of an \acf{AP}]{
				\input{./figures/curveTA.tex}
				\label{fig:actionPotential}
		}
		\end{minipage}
	}
	\framebox[0.50\textwidth]{
		\begin{minipage}{0.5\textwidth}
		\subfigure[The four stages of an \acf{AP}]{
			%\framebox[0.50\textwidth]{
				
\begin{tikzpicture}[->,>=stealth',shorten >=1pt,auto,
node distance=4cm,
semithick,scale=0.8, transform shape]
\tikzstyle{every state}=[rectangle,rounded corners,
minimum height = 1.9cm, text width=3.35cm, text centered, fill=blue!20,draw=none,text=black, draw,line width=0.3mm]
\tikzset{
	atrialCell/.style={
		fill=green!20,
		circle,
		minimum size=1.5cm,
		draw,
		line width=0.2mm
	}
}
\tikzset{
	path/.style={
		fill=green!20,
		rectangle,
		minimum size=1.5cm,
		draw,
		line width=0.2mm
	}
}


\node[state, 
label={[shift={(0,0)}]$v < V_T \wedge g(\vec{v_{I}}) < V_T$}, 
label={[shift={(-1.5,-2.5)}]\large $ \mathbf{q_0}$ }]
(Q0)  {$\dot{v_x}=C_1 v_x$ \\ $\dot{v_y}=C_2 v_y$ \\ $\dot{v_z}=C_3 v_z$ \\ $v=v_x - v_y + v_z$};

%entry
\path[<-, dashed] (Q0.150) edge node[below, align=left, shift={(1.15,1)}] {
	\footnotesize initial \\
	\footnotesize $\begin{matrix}
	v_{x}=0 \wedge v_{y}=0 \\
	\quad \wedge v_{z}=0 \wedge \theta=0
	\end{matrix}$
} ++(0cm,1.25cm);




\node[state, 
label={[shift={(0,0)}]$v < V_T \wedge g(\vec{v_{I}}) > 0$}, 
label={[shift={(-1.5,-2.5)}]\large $\mathbf{q_1}$   }]
(Q1) [node distance=6.3cm, right of=Q0] {$\dot{v_x}=C_4 v_x + C_7 g(\vec{v_{I}})$ \\ $\dot{v_y}=C_5 v_y + C_8 g(\vec{v_{I}})$ \\ $\dot{v_z}=C_6 v_z + C_9 g(\vec{v_{I}})$ \\ $v=v_x - v_y + v_z$};

\node[state, 
label={[shift={(0.2,-2.5)}]$v < V_O - 80.1 \sqrt{\theta}$}, 
label={[shift={(-1.5,-2.5)}]\large $\mathbf{q_2}$  }] 
(Q2) [below of=Q1] {$\dot{v_x}=C_{10} v_x$ \\ $\dot{v_y}=C_{11} v_y$ \\ $\dot{v_z}=C_{12} v_z$ \\ $v=v_x - v_y + v_z$};

\node[state, 
label={[shift={(0,-2.5)}]$v > V_R$}, 
label={[shift={(-1.5,-2.5)}]\large $\mathbf{q_3}$  }]
(Q3) [below of=Q0] {$\dot{v_x}=C_{13} v_x f(\theta)$ \\ $\dot{v_y}=C_{14} v_y f(\theta)$ \\ $\dot{v_z}=C_{15} v_z$ \\ $v=v_x - v_y + v_z$};

\path[->] (Q0.10) edge node[align=center] {
	\footnotesize $\{\tau\}$, \\
	\footnotesize $\{g(\vec{v_{I}}) \geq V_T\}$, \\
	\footnotesize $\left\{\begin{matrix} v^\prime_x = 0.3v \\ v^\prime_y = 0.0v \\ v^\prime_z = 0.7v \\ \theta^\prime = v / V_T \end{matrix}\right\}$
} (Q1.170);

\path[->] (Q1.190) edge node[align=center] {
	\footnotesize $\{\tau\}$, \\
	\footnotesize $\{g(\vec{v_{I}}) \leq 0 \wedge v < V_T$\}, \\
	\footnotesize $\left\{\begin{matrix} v^\prime_x = v_x \\ v^\prime_y = v_y \\ v^\prime_z = v_z \end{matrix}\right\}$
} (Q0.350);

\path[->] (Q1) edge node[align=center] {
	\footnotesize $\{\tau\}$, \\
	\footnotesize $\{v \geq V_T\}$, \\
	\footnotesize $\left\{\begin{matrix} v^\prime_x = v_x \\ v^\prime_y = v_y \\ v^\prime_z = v_z \end{matrix}\right\}$
} (Q2);

\path[->] (Q2) edge node[align=center] {
	\footnotesize $\{\tau\}$, \\
	\footnotesize $\{v \geq V_O - 80.1\sqrt{\theta}\}$, \\
	\footnotesize $\left\{\begin{matrix} v^\prime_x = v_x \\ v^\prime_y = v_y \\ v^\prime_z = v_z \end{matrix}\right\}$
} (Q3);

\path[->] (Q3) edge node[align=center, shift={(1.8,0)}] {
	\footnotesize $\{\tau\}$, \\
	\footnotesize $\{v \leq V_R\}$, \\
	\footnotesize $\left\{\begin{matrix} v^\prime_x = v_x \\ v^\prime_y = v_y \\ v^\prime_z = v_z \end{matrix}\right\}$
} (Q0);


\node[ draw,   inner xsep=1.1cm, inner ysep=1.9cm, shift={(0, 0.35)},
label={[label distance=0cm]60:~}, 
fit= (Q0)(Q2)] (BOX) {};

%--------------------------- nodes on top ------------

%label={[shift={(1,-0.4)}]\large $v$ },
%label={[shift={(1.1,-1.2)}]\large $v_{I_{0}}$ }]
%--------------------------- nodes on top ------------
\node[atrialCell, above of = Q0, xshift=-0.5cm, node distance = 5cm,
label={[shift={(1.8,-0.4)}]\large $v$ },
label={[shift={(1.8,-1.2)}]\large $v_{I_{0}}$ }]
(AN1){$N_1$};


\node[path, right of = AN1, node distance = 3.7cm] 
(P1){$Path$};

\node[atrialCell, right of = P1, node distance = 3.7cm,
label={[shift={(-1.8,-0.4)}]\large $v$ },
label={[shift={(-1.8,-1.2)}]\large $v_{I_{0}}$ }] 
(AN2){$N_2$};

\draw[<-](P1.160) -- (P1.160-|AN1.east);
\draw[->](P1.210) -- (P1.210-|AN1.east);

\draw[<-](P1.20) -- (P1.20-|AN2.west);
\draw[->](P1.330) -- (P1.330-|AN2.west);



\draw[-,dashed](AN1.-45) -- (BOX.north east);
\draw[-,dashed](AN1.230) -- (BOX.north west);




\end{tikzpicture}
				\label{fig:actionPotential}
			%}
		}
		\subfigure[The four stages of an \acf{AP}]{
			%\framebox[0.50\textwidth]{
				\input{./figures/curveHIOA.tex}
				\label{fig:actionPotential}
		%	}
		}
		\end{minipage}
	}

	
}
\end{figure*}

\begin{table*}
	\centering
	\renewcommand{\arraystretch}{1.3}
	\caption{Detailed qualitative comparison of heart models.}
	\label{tab:heart:uoa_oxford}
	
	\begin{tabular}{| L{2.5cm} | C{6cm} | C{6cm} |}
		\cline{2-3}
		\multicolumn{1}{L{1.5cm}|}{}
		& {\bf Timed Automata} 
		& {\bf Hybrid Input Output Automata} 
		\\\hline
		{\bf Heart Anatomy} (Figure~\ref{fig:heartNetwork}) 						
		& \multicolumn{2}{c|}{Electrical conduction system is modelled as a network of nodes connected by paths} 	 											
		\\ \hline
		{\bf Model of the node} 	
		&	Only captures the timing behaviours i.e., the time spent in each location
		&   Captures both the time and the voltage in each location. This captures the morphology of action potential. 		
		\\ \hline
		{\bf Model of the path} 	
		& Uses events to capture the behaviour of neighbouring cells.
		& Uses continuous values (voltage) to capture the behaviour of neighbouring cells. 	
		\\ \hline
		{\bf Formal verification} 	
		& Can be formally verified using existing TA tools such as UPPAAL.
		& Cannot be formally verified using existing tools.
		\\ \hline
		{\bf \acp{EGM}} 	
		& Due to the abstraction to \ac{TA}, only synthetic \acp{EGM} can be generated
		& Due to the continuous model, realistic  \acp{EGM} can be generated.
		\\ \hline
		{\bf Verification of Pacemaker (Figure~X) }	
		& Closed-loop verification with synthetic \acp{EGM}.
		& Not possible with existing tools.
		\\ \hline
	\end{tabular}
\end{table*}
